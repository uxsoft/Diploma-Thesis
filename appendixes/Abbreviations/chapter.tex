\chapter{Seznam použitých zkratek}
\label{ch:abbreviations}

\begin{description}
    \item[.NET] Framework vyvíjený firmou Microsoft
    \item[HTTP] HyperText Transfer Protocol 
    \item[GET] Dotazovací metoda protokolu HTTP
    \item[POST] Metoda protokolu HTTP
    \item[HTML] HyperText Markup Language jazyk webových stránek
    \item[MVVM] Softwarová architektura Model-View-View Model
    \item[REST] Representational State Transfer, architektura API založené na protokolu HTTP
    \item[XF] Xamarin.Forms, UI framework vyvíjený firmou Xamarin
    \item[UI] User Interface, uživatelské rozhraní
    \item[UX] User Experience, dojem uživatele ze zacházení s aplikací
    \item[API] Application Programming Interface, sbírka funkcí a tříd, kterou mohou programátoři využívat
    \item[JSON] JavaScript Object Notation, způsob zápisu objektů využívaný jazykem Javascript
    \item[SDK] Software Developer Kit, soubor nástrojů pro vývoj software
    \item[BaaS] Backend-as-a-Service, nabízení webových služeb pro aplikaci jako produkt
    \item[XML] eXtensible Markup Language obecný značkovací jazyk, který byl vyvinut a standardizován konsorciem W3C
    \item[XAML] eXtensible Application Markup Language, jazyk vyvinutý firmou Microsoft pro popis a initializaci struktury objektů
    \item[VPS] Virtual Private Server, je server běžící na virtualizovaném hardware provozovaný hostingovou službou
    \item[IL] intermediate language, je mezijazyk využívaný kompilátorem před generováním cílového strojového kódu
    \item[JNI] Java Native Interface je rozhraní umožňující propojit kód běžící na virtuálním stroji Javy s nativními programy a knihovnami napsanými v jiných jazycích
    \item[JIT] speciální metoda překladu do strojového kódu až za běhu programu
    \item[CSS] Cascading Style Sheets jsou jazyk pro popis způsobu zobrazení elementů na stránkách napsaných v jazycích HTML, XHTML nebo XML
    \item[WPF] Windows Presentation Foundation je UI framework součástí .NET
    \item[SQL] Structured Query Language, standardizovaný strukturovaný dotazovací jazyk, který je používán pro práci s daty v relačních databázích
    \item[IDE] Integrated Development Environment, software pro vývoj software.
    \item[ID] Identifikátor
    \item[IP] Internet Protocol, základním protokol pracující na síťové vrstvě používaný v počítačových sítích a Internetu
    \item[TFS] Team Foundation Server, centralizovaný systém správy verzí
    \item[Wi-Fi] Wireless Fidelity - označení pro několik standardů IEEE 802.11 popisujících bezdrátovou komunikaci v počítačových sítích
    \item[BlueTooth] je proprietární otevřený standard pro bezdrátovou komunikaci propojující dvě a více elektronických zařízení
\end{description}