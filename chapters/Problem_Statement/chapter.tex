\chapter{Problem Statement}
\label{ch:problem_statement}
\graphicspath{ {chapters/Problem_Statement/} }

The optimization problem of a robotic cell can be defined as follows.
There is a set of robots $R = \{ 1, \dots, m\} $ and a graph a graph $ G = (V, E, C) $ where its vertices V are operations (welding, moving, painting, waiting \dots
) and edges represent relations between the operations. \\

%(D_{min}, D_{max}, r)$
Let $V = \{1, \dots, n\}$, then every operation $i \in V$ is characteristic by its minimum ($\underline{d}_i$) and maximum($\overline{d}_i$) duration.
Additionally, for each robot $r \in R$ I define a set $O_r$, so that it contains all the operations assigned to it.
Each operation is assigned to exactly one robot, and this assignment means that the robot will execute the operation.\\

Edges $(i, j)$ in the edge set $E$ represent the precedences between operations. 
Like vertices, an edge $(i, j)$ also has several properties of its own. 
Mainly, we recognize three different types of edges defined in $type_{i,j}$. 
A \emph{robot loop} precedence which stems from the sequential order of the operations in the robots schedule, a \emph{link} which sets precedences between two operations of different robots and last but not least a \emph{robot loop reset} which starts the next cycle. 
Another property of an edge is the delay $D_{i,j} \geq 0$ which is specifying that the following operation $v_{to}$ can start no earlier than $D_{i,j}$ seconds after $i$ has completed in the same cycle. \\

Finally, there is the collision set $C$ which contains the pairs of operations $(i, j) \in V^2 : i \neq j$ that can't be executed at the same time, as this would result in a collision. \\

The goal of the optimization algorithm included in this work is to minimize cycle time $\omega$ as defined it in Chapter~\ref{ch:motivation}. \\

For example the robotic cell in Figure~\ref{fig:examplegraph} has only 2 robots and consists of operations $V = v_1, \dots, v_6$, with robots having 3 operations each. 
Robot $r_1$ has operations $v_1$ (move around left corner), $v_2$ (weld point 1), $v_3$ (weld point 2). 
Naturally these three operations need to be performed in a sequence, which is ensured by two  \emph{robot loop} edges which are marked red. 
When the robot finishes with operation $v_3$ he can start working on the next product coming on the assembly line hence the green \emph{robot loop reset} edge from $v_3$ to $v_1$.
Likewise for robot $r_2$. 
Figure~\ref{fig:examplegraph} also shows a \emph{link} edge which is highlighted blue and shows that $v_2$ must be finished before $v_5$ starts.
Let's say that the areas where $v_3$ and $v_6$ operate overlap (the points that are being spot-welded are too close together) and they can't be processed at the same time.
A new collision $c \in C$ would be introduced in the graph so that $c = (v_3, v_6)$.

\begin{figure}[H]
    \centering
    \includesvg[scale=0.6]{graph}
    \caption{Graph generated from a simple robotic cell}
    \label{fig:examplegraph}
\end{figure}
