\chapter{Experiments}

Osnova detailnejsi, hlidat si navaznost v textu - v teto kapitole popiseme algoritmus ktery jsme definovali v minule kapitole, vyvarovat se citovani wikipedie:D




To verify the validity of the proposed optimization model 30 problem instances were
generated, each of them corresponds to a robotic cell with 5 co-operating robots
where each robot has up to 3 power saving modes (motors, brakes, bus power off).

From 1 to 4 robot configurations are considered for each static activity and in aver-
age there are approximately 150 activities per instance. The production cycle time

is got by multiplying a lower bound by a factor from 1.05 to 1.40.

The energy optimization problem was formulated as an Integer Linear Program-
ming problem and solved by using IBM Ilog Cplex 12.6. Gentoo Linux server

equipped with 2 x Intel Xeon E5-2620 v2 @ 2.10 GHz processors and 64 GB mem-
ory was used for benchmarks.

As the first experiment the influence of Cplex time limit on quality of solutions
was investigated as shown in Table 4. It was found out that if the solver is given
2 hours instead of 100 seconds the quality of solutions improves about 3.3 % in
average. An average gap, which is a relative distance between the best found upper
bound (the best solution) and the lower bound, was 27.5 % for the two-hour limit.
The size of the model was roughly about 10000 constraints and 1000 variables.


LP SOLVE

\begin{center}
\begin{tabular}{|c|c|c|c|}
    \hline
    Complexity & Constraints & Variables & Time \\
    \hline
    2 & 235 & 231 & 20ms \\
3 & 332 & 335 & 17ms \\
4 & 449 & 464 & 27ms \\
5 & 566 & 593 & 33ms \\
6 & 708 & 752 & 48ms \\
7 & 892 & 963 & 71ms \\
8 & 1081 & 1179 & 98ms \\
9 & 1298 & 1431 & 134ms \\
10 & 1477 & 1635 & 170ms \\
11 & 1750 & 1954 & 229ms \\
12 & 1950 & 2184 & 284ms \\
13 & 2318 & 2621 & 383ms \\
14 & 2498 & 2825 & 434ms \\
15 & 2849 & 3241 & 549ms \\
16 & 3224 & 3684 & 690ms \\
17 & 3522 & 4034 & 814ms \\
18 & 3812 & 4374 & 956ms \\
19 & 4349 & 5017 & 1241ms \\
20 & 4730 & 5468 & 1448ms \\
21 & 5153 & 5971 & 1680ms \\
22 & 5422 & 6286 & 1876ms \\
23 & 6054 & 7044 & 2327ms \\
24 & 6475 & 7544 & 2639ms \\
25 & 6995 & 8166 & 3032ms \\
26 & 7261 & 8476 & 3269ms \\
27 & 7711 & 9012 & 3713ms \\
28 & 8228 & 9629 & 4267ms \\
29 & 9100 & 10681 & 5305ms \\
30 & 9662 & 11354 & 6532ms \\
31 & 10156 & 11943 & 7787ms \\
32 & 10550 & 12410 & 9359ms \\
33 & 11578 & 13652 & 12013ms \\
34 & 11743 & 13841 & 12451ms \\
35 & 12574 & 14842 & 13046ms \\
36 & 12960 & 15299 & 14080ms \\
37 & 13810 & 16323 & 15514ms \\
38 & 14865 & 17599 & 19106ms \\
39 & 15670 & 18569 & 21427ms \\
40 & 16136 & 19124 & 21700ms \\

    \hline
\end{tabular}
\end{center}