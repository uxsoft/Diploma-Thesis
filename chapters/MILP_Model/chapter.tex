\chapter{MILP Model}
\label{ch:milp_model}
\graphicspath{{chapters/MILP_Model/}}

The input to the algorithm is a graph \(G=(V, E, C)\) where \(V\) is a collection of vertices, $E$ is a collection of edges, and $C$ is a set of collision pairs of vertices. 
These pairs could cause collisions if executed simultaneously, as described in the previous chapter. 
For the purposes of the MILP model, we define the following variables. 
We want to minimize cycle time $\omega$. 
For operation $i \in V$ we define $s_i$ as the start of the operation.
Furthermore, $s'_i$ is a start time of the same operation with respect to cycle time $\omega$. 
The relation between $s_i$ and $s'_i$ is given by Equation~\ref{eq:milp_vertices} where $q_i$ is the index of the execution period. \\

Equation~\ref{eq:milp_edges} specifies precedences of the operations. 
The duration is split into the proposed duration $d_i$, and the wait time $d^w_i$. 
The wait time specifies how long the robot waits after the operation is finished before starting to work on the next operation.
The duration $d_i$ is constrained by its lower bound $\underline{d}_i$ as well as its upper bound $\overline{d}_i$. 
In this equation $h_{ij} = 1$ if the edge is a \emph{robot loop reset}, $h_{ij} = 0$ otherwise. \\

Equation~\ref{eq:milp_collisions} specifies the collision constrains. 
In this model, if there is a collision between operation A and operation B, either A ends before B starts or B ends before A starts, but they can't be running simultaneously. \\

This is a cyclic scheduling problem with resource constraints, which in our case are the collision zones.
This problem is NP-Hard, as shown by Hanen and Munier \cite{HanenMunierNPHardProof}. \\

\begin{align}
    \min_\omega\text{ } & \omega  \\
    \text{s.t. } \\
        & s_i  =  s'_i + q_i \omega & & \forall i \in V     \label{eq:milp_vertices}\\
        & s_i + d_i + d^w_i =  s_j + h_{ij} \omega & & \forall i, j \in E \label{eq:milp_edges}\\
        & s'_i + d_i  \leq  s'_j + x_{ij} \omega & & \forall i, j \in C \label{eq:milp_collisions}\\
        & x_{ij} + x_{ji}  =  1 & & \forall i, j \in C \\
        & \underline{d}_i \leq  d_i \leq \overline{d}_i & & \forall i \in V \\
    \text{where:} \\
        & \omega \in \mathbb{R}^+\\
        & s_i, s'_i, d_i \in \mathbb{R}^+\\
        & q_i \in \mathbb{Z}^+\\
        & x_{ij}, h_{ij} \in \{1, 0\} 
\end{align}

This model has many problems, mainly the multiplication of $x_{ij}$ and  $\omega$, therefore we use substitution to remove this multiplication of two variables and simplify the model. In the next step the following substitution was applied: $\tau = \frac{1}{\omega}$. \\

\begin{align}
\max_\tau\text{ } & \tau  \\
    \text{s.t. } \\
        & s_i \tau = s'_i \tau + q_i & & \forall i \in V \\
        & s_i \tau + d_i \tau + d^w_i \tau = s_j \tau + h_{ij} & & \forall i, j \in E \\
        & s'_i \tau + d_i \tau \leq s'_j \tau + x_{ij} & & \forall i, j \in C \\
        & x_{ij} + x_{ji} = 1 & & \forall i, j \in C \\
        & \underline{d}_i \tau \leq  d_i \tau \leq \overline{d}_i \tau & & \forall i \in V \\
    \text{where:} \\
        & \tau \in \mathbb{R}^+\\
        & s_i, s'_i, d_i \in \mathbb{R}^+\\
        & \tau \in \mathbb{R}^+ \\
        & q_i \in \mathbb{Z}^+\\
        & x_{ij}, h_{ij} \in \{1, 0\} 
\end{align}

And finally, after the last substitution $ S_i = s_i \tau $, $ S'_i = s'_i \tau $, $ D_i = d_i \tau $, $D^w_i = d^w_i \tau$ the following is what is implemented in the plug-in. \\

\begin{align}
    \displaystyle \max_\tau\text{ } & \tau  \\
    \textrm{s.t. } \\
        & S_i = S'_i + q_i & & \forall i \in V \\
        & S_i + D_i + D^w_i = S_j + h_{ij} & & \forall i, j \in E \\
        & S'_i + D_i \leq S'_j + x_{ij} & & \forall i, j \in C \\
        & x_{ij} + x_{ji} = 1 & & \forall i, j \in C \\
        & \underline{d}_i \tau \leq D_i \leq \overline{d}_i \tau & & \forall i \in V \\
    \text{where:} \\
        & S_i, S'_i, D_i \in \mathbb{R}^+\\
        & \tau \in \mathbb{R}^+ \\
        & q_i \in \mathbb{Z}^+\\
        & x_{ij}, h_{ij} \in \{1, 0\} 
\end{align}

At this stage, we have a linear MILP model which can be solved by most modern MILP solvers. In this final form, I used the MILP model in the optimization algorithm used in the plugin.\\