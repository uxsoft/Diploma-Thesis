\section{Intro}
Nowadays in the manufacturing industry most of the production is automated. Parts of the product are created and assembled in robotic cells or robotic assembly lines. When developing a new product the engineers usually create a CAD model of the product and based on that build prototypes. When refined a different engineering team is tasked with designing an assembly line that could mass produce the parts and assemble them together. \\ 

The assembly lines are designed in a different type of CAD software, which allows the simulation the full assembly sequence. This way the engineers can validate the reachability and collision clearance of the prepared line, including human factors. One example of such software is the Tecnomatix suite by SIEMENS. Tecnomatix Process Simulate is an industry leading software for digital manufacturing. \\ 

When designing an assembly line for a product the focus is on the successful creation of the product, which in itself is a no mean feat. Then searching for the optimal layout of the robots and schedule of the tasks that need to be executed for the desired result is an inhuman task. 

\subsection{Motivation}

Rather than focusing the question of the layout of the robots which would require knowing the available space, layout of the factory, location of the power outlets and so on. Since this problem would be bigger than one lifetime of work, we need to apply the divide and conquer principle. Rather than looking at the whole problem this thesis tries to tackle a smaller part, hoping that other research could build on it. The goal is to take the human design of the assembly line and apply a mathematical model to find the optimal schedule which could be used to marginally improve the effectivity of the line. \\

The schedule could be optimized with different goals in mind, for example: cycle time or energy consumption (leveraging the robots sleep modes). This work focuses on building the bridge between the design tool and a MILP model which can be swapped out. 

\subsection{Related Work}

strukturovane co se tyce algoritmu, toolu, atd\ldots vypichnuti dulezitych veci co vyuzije sekce Contribution)

\subsection{Contribution and Outline}

vzhledem k existujicim pracem neexistuje takova a takova prace + popis inovaci, a v pristi sekci bude ... a pak ... atd ...

\section{Problem Statement}
- Mejme n robotu a mejme graf co popisuje operace\ldots 