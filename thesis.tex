\documentclass[11pt,twoside,a4paper]{book}
\usepackage[utf8]{inputenc}
\usepackage[english]{babel}
\usepackage{caption}
\usepackage{subcaption}
\usepackage{minted}
\usepackage{amssymb}
\usepackage{mathtools}
\usepackage{listings}
\usepackage{graphicx}
\usepackage{hyperref}
\usepackage[dvipsnames,hyperref]{xcolor}
\usepackage{pdfpages}
\usepackage{caption}
\usepackage{csquotes}
\usepackage[style=numeric,citestyle=ieee,hyperref,natbib,backend=biber]{biblatex}
\addbibresource{reference.bib}

\definecolor{aliceblue}{RGB}{240,248,255}
\definecolor{wonderland}{RGB}{15,123,196}
\hypersetup{%
    colorlinks=true, linktocpage=true, pdfstartpage=1, 
    pdfstartview=FitV, breaklinks=true, pdfpagemode=UseNone, 
    pageanchor=true, pdfpagemode=UseOutlines,%
    plainpages=false, bookmarksnumbered,
    bookmarksopen=true,%
    bookmarksopenlevel=1,%
    hypertexnames=true, pdfhighlight=/O,%
    urlcolor=wonderland,
    linkcolor=RoyalBlue, 
    citecolor=wonderland,%
    hyperfootnotes=false,pdfpagelabels,
    pdfsubject={},%
    pdfkeywords={},%
    pdfcreator={pdfLaTeX},%
    pdfproducer={LaTeX}%
}

\let\tmp\oddsidemargin
\let\oddsidemargin\evensidemargin
\let\evensidemargin\tmp
\reversemarginpar

\usepackage{titlesec}
\usepackage{fancyhdr}
\usepackage{graphicx}
\usepackage{titling}
\usepackage{xstring}

\definecolor{darkaccent}{HTML}{007AC2}
\definecolor{lightaccent}{HTML}{107AC2}
\urlstyle{same}

% custom commands
\newcommand*{\captionsource}[2]{%
  \caption[{#1}]{%
    #1%
    \\\hspace{\linewidth}%
    \textbf{Source:} #2%
  }%
}

%page style
\pagestyle{fancy}
\renewcommand{\headrulewidth}{0pt}
\renewcommand{\footrulewidth}{0pt}
\setlength{\headheight}{14pt} 

%headers
\fancyhf{}
\fancyhead[LE,RO]{\leftmark}
\fancyhead[RE,LO]{}
\fancyfoot[CE,CO]{\thepage}


%section
\titleformat{\section}
[hang]
{\Large\bfseries}
{
\color{darkaccent}
\raisebox{-2pt}{
    \rule{0.35cm}{1em}
}
\color{black}
\sffamily
\thesection
}
{10pt}
{
\color{lightaccent}
\sffamily
\Large\bfseries
}
 
%subsection
\titleformat{\subsection}
[hang]
{\large\bfseries}
{
\color{darkaccent}
\raisebox{-2pt}{
    \rule{0.35cm}{1em}
}
\color{black}
\sffamily
\thesubsection
}
{10pt}
{
\color{lightaccent}
\sffamily
\large\bfseries
}

%subsubsection
\titleformat{\subsubsection}
[hang]
{\normalsize\bfseries}
{
    \empty
}
{10pt}
{
\color{darkaccent}
\raisebox{-2pt}{
    \rule{0.35cm}{1em}
}
\color{black}
\sffamily
\thesubsubsection
\color{lightaccent}
\hspace{1em}
\normalsize\bfseries
}
 
%chapter
\titleformat
{\chapter} 
[display] 
{\bfseries\huge} 
{
    \color{darkaccent}
    \raisebox{0em}{
        \rule{0.35cm}{1em}
    }
    \color{black}
    \sffamily
    \chaptertitlename \ \Huge\thechapter
} 
{-0.75em}
{
    \color{darkaccent}
    \raisebox{-2pt}{
        \rule{0.35cm}{1.5em}
    }
    \sffamily
    \color{lightaccent}
}

\renewcommand{\maketitle}
{
    \begingroup % Create the command for including the title page in the document
    \hbox
    { % Horizontal box
        \color{darkaccent}
        \rule{0.35cm}{\textheight} % Vertical line
        \color{black}
        \hspace*{0.05\textwidth} % Whitespace between the vertical line and title page text
        \parbox[b]{0.9\textwidth}
        { % Paragraph box which restricts text to less than the width of the page
            
            \noindent\sffamily\large\textbf{\TypeOfWork}\vspace{\baselineskip}\\
            
            \begin{tabular}{ll}
                \includegraphics[width=3cm]{ctulogo} & 
                
                \color{lightaccent}\sffamily\Large\textbf{
                    \parbox[b]{3cm}{
                        \University
                    }
                    \vspace{1em}
                } \\
                \Huge\color{lightaccent}\sffamily\textbf{F3} & 
                \hspace{0.2cm}\parbox[b]{8cm}{
                    \textbf{\Faculty} \\
                    \textbf{\Department}
                }
            \end{tabular}
            \vspace{3cm}
            
            \color{lightaccent}\sffamily\huge\textbf{\WorkTitle}\\
            
            \Large\color{black}\textbf{\Author} \\
            \normalsize\textbf{\StudyProgram} \\

            \vspace{6.5cm}
            
            \normalsize
            \textbf{\Date}\\
            \SupervisorLabel: \Supervisor
        }
    }
    \endgroup
}

\newcommand{\splitpage}[3]{
    \begingroup
    \begin{center}
    \color{lightaccent}\huge\sffamily\centering\textbf{#1}
    \end{center}
    \raisebox{0.83\textheight}{\parbox[t]{0.45\textwidth}{
        #2
    }}
    \color{darkaccent}
    \hspace{1mm}
    \rule{0.35cm}{0.85\textheight} % Vertical line
    \hspace{1mm}
    \color{black}
    \raisebox{0.83\textheight}{\parbox[t]{0.45\textwidth}{
        #3
    }}
    \endgroup
}

\newcommand{\class}[1]{
    \cls{#1}
}

\newcommand{\namespace}[1]{
    \cls{#1}
}

\newcommand{\cls}[1]{
    \StrSubstitute{#1}{/}{$\backslash$\-}[\clscontent]
    \StrSubstitute{\clscontent}{-}{\-}[\clscontent]
    \emph{\StrSubstitute{\clscontent}{.}{.\-}}
} %Template

\author{Jan Dryk}

\newcommand\Date{\today}
\newcommand\Department{Katedra počítačů}
\newcommand\Faculty{Fakulta elektrotechnická}
\newcommand\University{České vysoké učení technické v Praze}
\newcommand\labelSupervisor{Vedoucí práce}
\newcommand\labelStudProgram{Studijní program}
\newcommand\labelStudBranch{Obor}
\newcommand\TypeOfWork{Diplomová práce}
\newcommand\StudProgram{Otevřená informatika}
\newcommand\StudBranch{Softwarové inženýrství}
\newcommand\WorkTitle{An extension of  Process Simulate for optimization of robotic cells}
\newcommand\FirstandFamilyName{Jan Dryk}
\newcommand\Supervisor{Ph.D. Přemysl Šůcha}


\begin{document}
\pagenumbering{gobble}
%\includepdf{zadani.pdf}
\cleardoublepage

\pagenumbering{roman}
%Title page
\maketitle
\cleardoublepage
%end Title page

%Declaration and Acknowledgement
\splitpage{Poděkování / Prohlášení}
{% Acknowledgement
	V první řadě bych chtěl poděkovat svému vedoucímu, Ph.D. Přemyslu Šůchovi, za poskytnutí konzultací a cenných připomínek.s
	Dále bych chtěl poděkovat své rodině za podporu při studiu i behěm psaní této práce.
	V poslední řade děkuji za spolupráci a rady všem členům týmu.
}
{% Declaration
	Prohlašuji, že jsem práci vypracoval samostatně a použil jsem pouze podklady uvedené v~přiloženém seznamu.\\
	Nemám závažný důvod proti užití tohoto školního díla ve smyslu \S 60 Zákona č.~121/2000 Sb., o právu autorském, o právech souvisejících s právem autorským a o změně některých zákonů (autorský zákon).
\\[15mm]
	V~Praze dne \today \vspace{10mm} \hfill \hbox to 58mm{\tiny\dotfill}
}

\cleardoublepage
%end Declaration and Acknowledgement


\splitpage{Abstrakt / Abstract}
{
    Process Simulate od firmy Siemens je průmyslovým standardem v oblasti softwaru pro návrh výrobních linek.
Umožňuje výrobcům plánovat a ověřovat montážní linku dlouho než začneme stavět budovu.
Tato práce usiluje o to, aby uživatelům pomohla zlepšit kvalitu svých návrhů tím, že jim poskytne rozšíření aplikace zaměřené na optimalizaci.
Toto rozšíření poskytuje uživatelské rozhraní k zahrnutému optimalizačnímu algoritmu, jehož cílem je minimalizovat dobu cyklu a zároveň zabránit kolizím.
Nejprve robotickou buňku a operaci analyzuje a poté upravuje operaci podle optimálního řešení, které našel optimizační algoritmus.
Řešení bylo navrhnuto modulárně, aby mohlo být v budoucnu rozšířeno o sofistikovanější optimalizační algoritmy.
}
{
    Process Simulate by Siemens is the industry standard in the area of assembly line design software. 
It allows the manufacturers to plan and validate an assembly line long before breaking the ground.
This work strives to help the users improve the quality of their designs by providing them with an optimization plugin.
This plugin provides a user interface to the included optimization algorithm which, aims to minimize the cycle time while avoiding any collisions. 
First, it analyzes the robotic cell and the operation and then it adjusts the operation according to the optimal solution found by the algorithm. 
The plugin was built with modularity in mind so that it can be extended with more sophisticated optimization algorithms in the future.
}

\pagenumbering{arabic}\setcounter{page}{1}

\tableofcontents

%CHAPTERS

\section{Intro}
Nowadays in the manufacturing industry most of the production is automated. Parts are created in so called robotic cells. A robotic cell is basically an assembly line with robots. In the center of the cell would be typically a conveyor belt or a robot which is tasked by moving the part around, facilitating   When developing a new product . This way 

The assembly lines are designed by humans in a CAD software. One example of such software is the Tecnomatix suite by SIEMENS.

\subsection{Motivation}
The goal of this work is to take the human design and using a mathematical model of the assembly line to find the optimal schedule


- Related Work (strukturovane co se tyce algoritmu, toolu, atd... vypichnuti dulezitych veci co vyuzije sekce Contribution)

- Contribution and Outline (vzhledem k existujicim pracem neexistuje takova a takova prace + popis inovaci, a v pristi sekci bude ... a pak ... atd ...)

\section{Problem Statement}
- Mejme n robotu a mejme graf co popisuje operace... 

\section{Algorithm}
Input to the algorithm is a graph $G=(V, E, C)$ where $V$ is a collection of vertices, $E$ is a collection of edges and $C$ is a collection of pairs of edges that 
Each edge has a minimum duration dMin and 

\begin{equation}
\begin{matrix}
\displaystyle \min_\omega & \omega  \\
\textrm{s.t.} \\
& s_i & = & s'_i + q_i \omega & & \forall i \in V \\
& s_i + d_i & = & s_j + h_{ij} \omega & & \forall i, j \in E \\
& s'_i + d_i & \leq & s'_j + x_{ij} \omega & & \forall i, j \in C \\
& x_{ij} + x_{ji} & = & 1 & & \forall i, j \in C \\
& \underline{d}_i \leq &  d_i & \leq \overline{d}_i & & \forall i \in V \\

\textrm{where:} \\
& \omega \in \mathbb{R}^+\\
& s_i, s'_i, d_i \in \mathbb{R}^+\\
& q_i \in \mathbb{Z}^+\\
& x_{ij}, h_{ij} \in \{1, 0\} \\

\end{matrix}
\end{equation}

This model has many problems, mainly the multiplication of $x_{ij}$ and  $\omega$, therefore we use substitution to remove this multiplication of two variables and simplify the model. In the next step the following substitution was applied: $\tau = \frac{1}{\omega}$

\begin{equation}
\begin{matrix}
\displaystyle \max_\tau & \tau  \\
\textrm{s.t.} \\
& s_i \tau & = & s'_i \tau + q_i & & \forall i \in V \\
& s_i \tau + d_i \tau & = & s_j \tau + h_{ij} & & \forall i, j \in E \\
& s'_i \tau + d_i \tau & \leq & s'_j \tau + x_{ij} & & \forall i, j \in C \\
& x_{ij} + x_{ji} & = & 1 & & \forall i, j \in C \\
& \underline{d}_i \tau \leq &  d_i \tau & \leq \overline{d}_i \tau & & \forall i \in V \\

\textrm{where:} \\
& \tau \in \mathbb{R}^+\\
& s_i, s'_i, d_i \in \mathbb{R}^+\\
& \tau \in \mathbb{R}^+ \\
& q_i \in \mathbb{Z}^+\\
& x_{ij}, h_{ij} \in \{1, 0\} \\

\end{matrix}
\end{equation}

And finally, after the last substitution $ S_i = s_i \tau $, $ S'_i = s'_i \tau $, $ D_i = d_i \tau $ the following is what is implemented in the plug-in.

\begin{equation}
\begin{matrix}
\displaystyle \max_\tau & \tau  \\
\textrm{s.t.} \\
& S_i & = & S'_i + q_i & & \forall i \in V \\
& S_i + D_i & = & S_j + h_{ij} & & \forall i, j \in E \\
& S'_i + D_i & \leq & S'_j + x_{ij} & & \forall i, j \in C \\
& x_{ij} + x_{ji} & = & 1 & & \forall i, j \in C \\
& \underline{d}_i \tau \leq &  D_i & \leq \overline{d}_i \tau & & \forall i \in V \\

\textrm{where:} \\
& S_i, S'_i, D_i \in \mathbb{R}^+\\
& \tau \in \mathbb{R}^+ \\
& q_i \in \mathbb{Z}^+\\
& x_{ij}, h_{ij} \in \{1, 0\} \\

\end{matrix}
\end{equation}
- ILP Model


\section{Integration with Process Simulate}
- Dev environment

- Modules Classes and their Responsibilities

- Extension the selection of optimization providers

\section{Experiments}
(Osnova detailnejsi, hlidat si navaznost v textu - v teto kapitole popiseme algoritmus ktery jsme definovali v minule kapitole, vyvarovat se citovani wikipedie:D, )
- 

\section{Conclusion}
- 

\section{Literature}
(zacit sbirat zdroje co nejdrive)

\chapter{Algorithms}
\label{ch:algorithms}
\graphicspath{{chapters/Algorithms/}}

\section{Algorithm}
Input to the algorithm is a graph \(G=(V, E, C)\) where \(V\) is a collection of vertices, $E$ is a collection of edges and $C$ is a collection of pairs of edges that 
Each edge has a minimum duration dMin and 

\begin{equation}
\begin{matrix}
\displaystyle \min_\omega & \omega  \\
\textrm{s.t.} \\
& s_i & = & s'_i + q_i \omega & & \forall i \in V \\
& s_i + d_i & = & s_j + h_{ij} \omega & & \forall i, j \in E \\
& s'_i + d_i & \leq & s'_j + x_{ij} \omega & & \forall i, j \in C \\
& x_{ij} + x_{ji} & = & 1 & & \forall i, j \in C \\
& \underline{d}_i \leq &  d_i & \leq \overline{d}_i & & \forall i \in V \\

\textrm{where:} \\
& \omega \in \mathbb{R}^+\\
& s_i, s'_i, d_i \in \mathbb{R}^+\\
& q_i \in \mathbb{Z}^+\\
& x_{ij}, h_{ij} \in \{1, 0\} \\

\end{matrix}
\end{equation}

This model has many problems, mainly the multiplication of $x_{ij}$ and  $\omega$, therefore we use substitution to remove this multiplication of two variables and simplify the model. In the next step the following substitution was applied: $\tau = \frac{1}{\omega}$

\begin{equation}
\begin{matrix}
\displaystyle \max_\tau & \tau  \\
\textrm{s.t.} \\
& s_i \tau & = & s'_i \tau + q_i & & \forall i \in V \\
& s_i \tau + d_i \tau & = & s_j \tau + h_{ij} & & \forall i, j \in E \\
& s'_i \tau + d_i \tau & \leq & s'_j \tau + x_{ij} & & \forall i, j \in C \\
& x_{ij} + x_{ji} & = & 1 & & \forall i, j \in C \\
& \underline{d}_i \tau \leq &  d_i \tau & \leq \overline{d}_i \tau & & \forall i \in V \\

\textrm{where:} \\
& \tau \in \mathbb{R}^+\\
& s_i, s'_i, d_i \in \mathbb{R}^+\\
& \tau \in \mathbb{R}^+ \\
& q_i \in \mathbb{Z}^+\\
& x_{ij}, h_{ij} \in \{1, 0\} \\

\end{matrix}
\end{equation}

And finally, after the last substitution $ S_i = s_i \tau $, $ S'_i = s'_i \tau $, $ D_i = d_i \tau $ the following is what is implemented in the plug-in.

\begin{equation}
\begin{matrix}
\displaystyle \max_\tau & \tau  \\
\textrm{s.t.} \\
& S_i & = & S'_i + q_i & & \forall i \in V \\
& S_i + D_i & = & S_j + h_{ij} & & \forall i, j \in E \\
& S'_i + D_i & \leq & S'_j + x_{ij} & & \forall i, j \in C \\
& x_{ij} + x_{ji} & = & 1 & & \forall i, j \in C \\
& \underline{d}_i \tau \leq &  D_i & \leq \overline{d}_i \tau & & \forall i \in V \\

\textrm{where:} \\
& S_i, S'_i, D_i \in \mathbb{R}^+\\
& \tau \in \mathbb{R}^+ \\
& q_i \in \mathbb{Z}^+\\
& x_{ij}, h_{ij} \in \{1, 0\} \\

\end{matrix}
\end{equation}
- ILP Model


\printbibliography
\appendix


\chapter{Abbreviations}
\label{ch:abbreviations}

\begin{description}
    \item[.NET] Framework for writing applications developed by Microsoft
    \item[CAD] Computer Assisted Design
    \item[API] Application Programming Interface
    \item[MVVM] Model–View–ViewModel architectural pattern
    \item[WPF] Windows Presentation Foundation
    \item[XAML] Extensible Application Markup Language
    \item[DI] Dependency Injection
    \item[IOC] Inversion of Control
    \item[MILP] Mixed-Integer Linear Programming
\end{description}
\chapter{Manual}
\label{ch:manual}
\graphicspath{{appendixes/Manual/}}


\chapter{CD Contents}
\label{ch:cdcontents}

\begin{itemize}
\item \textbf{/Bin}
    \begin{itemize}
    \item \textbf{CashBob.Mobile.apk} - instalační balíček mobilní aplikace CashBob pro operační systém Android
    \item \textbf{CashBobServer.zip} - archiv s aplikací CashBob Server
    \item \textbf{CashBobClient.zip} - archiv s aplikací CashBob Client
    \end{itemize}
    
\item \textbf{/Source}
    \begin{itemize}
    \item \textbf{/CashBob.Mobile} - zdrojové kódy mobilních aplikací
    \item \textbf{/CashBob} - zdrojové kódy systému CashBob
    \end{itemize}
\item \textbf{/Doc}
    \begin{itemize}
    \item \textbf{/Design} - grafické produkty této práce
    \item \textbf{/Thesis} - zdrojové soubory textové části práce (\LaTeX)
    \item \textbf{/Documentation} - vygenerovaná HTML dokumentace k mobilní aplikaci
    \item \textbf{manual.pdf} - samostatná instalační a uživatelská příručka
    \item \textbf{drykjan\_2015bach.pdf} - text bakalářské práce ve formátu PDF
    \item \textbf{vykaz.xlsx} - Výkaz práce ve formátu aplikace  Microsoft Excel
    \end{itemize}

\item \textbf{/Video}
    \begin{itemize}
    \item \textbf{uitest.mp4} - Video-ukázka probíhajícího UI testu
    \end{itemize}
\item \textbf{readme.txt} - textový soubor s popisem obsahu CD
\end{itemize}

\end{document}
